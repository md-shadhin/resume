\documentclass[10pt,twocolumn]{article}

\usepackage{two-row-acad-cv}
\usepackage{hyperref}
\hypersetup{
    colorlinks=true,
    linkcolor=blue,
    filecolor=magenta,      
    urlcolor=blue,
    pdftitle={Sharelatex Example},
    pdfpagemode=FullScreen,
    }

\urlstyle{same}
\colorlet{cvthemecolour}{black}
\colorlet{cvcolour}{cyan}
\setcounter{secnumdepth}{0}
\thispagestyle{empty}

\begin{document}

\subsection{Md. Sakil Khan Shadhin}
\begin{tabular}{>{\itshape}r|l}
    Email & \href{mailto:shadhin.khan@northsouth.edu}{shadhin.khan@northsouth.edu} \\
    Phone & +8801747927935\\
    Address & Nakhalpara, Tejgaon\\&Dhaka, Bangladesh
\end{tabular}

\cvrule{black}{2pt}

\subsection{PROFESSIONAL LINKS}
\begin{itemize}
  \item \url{https://github.com/shadhinkhan}
  \item \url{https://scholar.google.com/citations?user=3B-SNn8AAAAJ}
  \item \url{https://www.researchgate.net/profile/Md\_Shadhin}
  \item \url{https://www.linkedin.com/in/mdsks/}
\end{itemize}

\cvrule{black}{2pt}

\subsection{SKILLS}
\begin{itemize}
  \item C, C++, Python, Java, Kotlin, PHP, JavaScript
  \item Django, CodeIgniter
  \item PostgreSql, MySql
\end{itemize}

\cvrule{black}{2pt}

\subsection{PROJECT}
\subsubsection{Clustering Students Based on Their Evaluations of
Teaching and Teachers}
\begin{tabular}{>{\bfseries}r >{\footnotesize}p{0.4\textwidth}}
   & Student evaluations have turned into the focal point
of broad information gathering, because of high levels of rivalry
in education. But the information disclosure isn’t sufficient,
particularly in the utilization of student evaluations. The purpose
of this study is to identify various clusters of students based on
their evaluations of teaching and teachers. The accompanying
conclusion will be drawn from this exploration that diverse
topological structures of student exist in the institution.
\end{tabular}

\cvrule{black}{2pt}


\subsection{WORK EXPERIENCE}
\begin{tabular}{r| p{0.7\textwidth}}
    \cvevent{2019--Present}{\href{https://www.brotecs.com/}{BroTecs Technologies Limited}}{Jr. Software Engineer}{Dhaka, BD}{}{}
\end{tabular}



\cvrule{black}{2pt}

\subsection{EDUCATION}
\begin{tabular}{r p{0.8\textwidth}}
    \cvdegree{2019}{\href{http://www.northsouth.edu/}{North South University}}{Computer Science \& Engineering}{Dhaka, BD}{}{}
\end{tabular}

\cvrule{black}{2pt}

\subsection{COMPETITION}
\subsubsection{\href{https://www.kaggle.com/c/competitive-data-science-final-project}{Predict future sales}}
\begin{tabular}{>{\bfseries}r >{\footnotesize}p{0.4\textwidth}}
   & I and my team "Outliers", we describe the methods and features in the Predict Future
Sale competition in Kaggle. The competition involved a time series problem to
solve for a renowned Russian software company, 1C. We had to predict one
month’s sale based on 34 months of sale of 20170 different items in over 60 shops
from different cities of Russia. We generated many features based on statistics,
texts, domain knowledge and intuition. The time difference between an item being
sold in each shop, total duration of an item staying in a shop, the opening and
closing of an item and a shop are among the total of 83 features. We also
developed many lag features based on month number, price, and revenue. City
code, type and sub types of items from the name of the items, one hot encoded
feature based on the number of days in a month, month number in a year along
with the given features in the competition were fed into a gradient boosting model
to train and generate the final month’s sale. These categorical and numerical
features gave the best result we found which placed Outliers team in 10th among
more than 2000 teams currently and the competition is still ongoing.
\end{tabular}

\cvrule{black}{2pt}

\subsection{PUBLICATION}
\begin{tabular}{>{\footnotesize\bfseries}r>{\footnotesize}p{0.4\textwidth}}
    & Nahian Ahmed, Nazmul Alam Diptu, \textbf{M. Sakil Khan Shadhin}, M. Abrar Fahim Jaki, M. Ferdous Hasan, M. Naimul Islam and Rashedur M. Rahman, \href{https://doi.org/10.1142/S2196888819500246}{"Artificial Neural Network and Machine Learning Based Methods for Population Estimation of Rohingya Refugees: Comparing Data-Driven and Satellite Image-Driven Approaches"}, Vietnam Journal of Computer Science, 2019
\end{tabular}

\end{document}